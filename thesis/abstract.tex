\section*{Abstract}
\addcontentsline{toc}{section}{\protect\numberline{}Abstract}

Cavity electrodynamics is generally the study of light confined inside an optical resonator, and is widely utilized across many fields for its ability to quantize electromagnetic fields in a basis of harmonic oscillator modes. We propose novel optical Fano microcavities based on pairs of suspended, ultrathin, resonant SiN membranes, patterned into sub-wavelength gratings, as mirrors. We denote this specific cavity the \emph{double Fano cavity}. The coupling of the guided-modes of the sub-wavelength gratings with the mode of the incident laser leads to narrow spectral profiles referred to as Fano resonances. Due to these Fano resonances, the double Fano cavity has shown to maintain high optical Q-factors, even for cavity lengths in the low micrometer regime, producing linewidths on a scale in the range of picometers.  

The double Fano cavity is investigated thoroughly both numerically and experimentally. The recorded spectra are compared with a modified Fabry-Perot transmission function with wavelength-dependent optical amplitude coefficients described thoroughly theoretically and verified by comparison with a simulated structure resembling a sub-wavelength grating. 

The results presented are structurally in great agreement with the theoretical predictions made, and while the stability of the resonance transmission profile is to be improved, the future potential use-cases for the double Fano cavity provides optimistic outlooks in terms of both research and applications. As an example hereof, the SiN membranes are recognized for having high mechanical Q-factors, and if patterned to similarly posses excellent optical capabilities, they are prime candidates for optomechanical investigations. Other applications are found in fields such as quantum optics, photonics or sensing. In this thesis, I attempt to provide the foundation for understandnig this versatile optical device, by a systematic investigation of the double Fano cavity.