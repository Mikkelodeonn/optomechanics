\section{Method}
\subsection{The experimental setup}

The experimental setup used to optically characterize the Fano mirrors, single- and double Fano cavities is illustrated in figure \ref{fig:experimental_setup} where the whole setup is sketched in figure \ref{fig:setup_sketch} and the specific part of the setup surrounding the cavity, outlined by the dashed line in figure \ref{fig:setup_sketch}, is subsequently shown in figure \ref{fig:setup_zoomed}. 

In order to effectively conduct the experiments in this project, it is imperative to be able to control certain parameters. Each element in the experimental setup is thoroughly considered for each their purpose in this regard, these will be outlined in this section. 

\begin{figure}[h!]
    \centering
    \begin{subfigure}[b]{0.4\textwidth}
        \includegraphics[width=\textwidth]{figures/setup_sketch.pdf}
        \caption{}
        \label{fig:setup_sketch}
    \end{subfigure}
    \hfill
    \begin{subfigure}[b]{0.59\textwidth}
        \includegraphics[width=\textwidth]{figures/setup_skecth_zoomed.pdf}
        \caption{}
        \label{fig:setup_zoomed}
    \end{subfigure}
    \caption{(a) shows schematics of the experimental setup for measuring cavity transmission- and Fano mirrors transmission/reflectivity spectra. The dashed line indicate the location of the cavity for measurements, and the setup is seen in (b). Note, the setup is shown with an optical cavity present, and is thus modified accordingly when prepared for Fano mirror measurements. The components not outlined in this section are regular mirrors labeled $M_{1-3}$ apatures used for alignment labeled $I_{1-4}$.}
    \label{fig:experimental_setup}
\end{figure}

\subsubsection{Tunable diode laser}

As shown in figure \ref{fig:experimental_setup} the laser source used for the optical characterizations is coupled into the setup through an optical fiber. The laser used is a \emph{Toptica DLC Pro} tunable CW diode laser with a range of $910$ $-$ $980nm$. The optical fiber is a \emph{P3-780PM-FC-10} fiber from Thorlabs which is a single mode\footnote{A single mode fiber can only sustain the TEM00 mode, which means that the output is known to be perfectly Gaussian.} polarization-maintaining optical fiber with an effective range of $770$ $-$ $1100nm$. Between the Toptica laser and the incoupling end of the fiber, a $\lambda/2$\emph{-plate} (HWP) and \emph{polarizing beam splitter} (PBS) is placed in order to be able to control the incident power of the laser and to only couple linearly polarized light into the setup.  

The light being emitted from the optical fiber is sent through another HWP and PBS in order to be able to control the resulting polarization in the setup even more precisely, should there be any discrepancies of the light coupled into the fiber. 

\subsubsection{$\lambda / 2$ - waveplate}

A $\lambda/2$-waveplate, or HWP, is constructed of a so-called bi-refringent material (most commonly crystaline quartz), which means that it has slightly different refractive indicies for incident light of different polarization axis'. Generally a HWP will have a \emph{fast}- and \emph{slow axis}, where it is understood that light polarized along the fast axis experiences a lower refractive index (and hence moves faster), than that along the slow axis. In this way the HWP separates the components of unpolarized light that has perpendicular and parallel polarizations with respect to the fast axis. 

\begin{figure}[h!]
    \centering
    \includegraphics[width=0.6\textwidth]{figures/HWP.pdf}
    \caption{A simple sketch of the effect of a HWP on linearly polarized light.}
    \label{fig:HWP}
\end{figure}

The effect of the HWP on linearly polarized light, is an effective rotation of the polarization, this is sketched simply in figure \ref{fig:HWP}. It can be shown that the polarization axis is rotated according to the angle between the fast axis of the HWP and the incident polarization axis. A relative angle $\theta$ will thus result in a rotation of $2\theta$, e.g. if $\theta=45^{\circ}$, this will constitute a rotation from completely p-polarized light to completely s-polarized light. This is the specific scenario sketched in figure \ref{fig:HWP}\cite{edmund_optics}. In this way a rotating HWP can allow one to alter an incident linearly polarized beam to be polarized along any axis, and is thus a necessary component for this particular setup.

\subsubsection{Optical telescope}

The linearly polarized light transmitted through the PBS passes through plano-convex lenses of positive focal lengths $f$ and $f^{\prime}$, $L_1$ and $L_2$, which together makes up an optical telescope used to manipulate the beam waist $w_0$ incident on the cavity or Fano mirror. 

Figure \ref{fig:telescope} shows schematics of the general way an optical telescope is utilized to manipulate the beam waist of a laser beam.

\begin{figure}[h!]
    \centering
    \includegraphics[width=0.5\textwidth]{figures/optical_telescope.pdf}
    \caption{}
    \label{fig:telescope}
\end{figure}

When the incident beam hits $L_1$ it is focused according to the focal length $f$ of this lens, and by inserting another lens $L_2$ of a relatively longer focal length one can \emph{catch} the beam at the desired beam waist. If the focal length $f^{\prime}$ of $L_2$ is sufficently long, compared to the path of the beam after the lenses, the beam will be approximately collimated and remain at the waist obtained when incident on $L_2$.

\subsubsection{Transmission, reflection and incident photo detectors}

After passing the optical telescope the beam reaches a simple 50/50 beam splitter (BS) which transmits 50\% of the light while reflecting another 50\%, which is then incident on the cavity or Fano mirror (target). 

The transmitted light passes through a lens $L_3$ which is focused on a detector $P_I$ used for reference measurements and later normalization. Since the tunable laser in nature varies in power with the wavelength, it is necesarry to keep track of these fluctuations and correct for them during data analysis. 

The transmitted light is sent through, yet another, HWP which in this case is used only to alter the polarization of the light incident on the target. After the beam, or a portion of it, has passed the target it is sent through a lens $L_5$ focused onto the transmission detector $P_T$.

The part of the light incident on the target that is \emph{not} transmitted, is reflected back onto the BS which then transmits 50\% once again, and thus reflects the other 50\%. The transmitted part here is focused through the lens $L_4$ onto the reflection detector $P_R$.
