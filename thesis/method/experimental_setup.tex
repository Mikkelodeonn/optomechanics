\section{Method}
\subsection{The experimental setup}\label{sec:experimental_setup}

The experimental setup used to optically characterize the Fano mirrors, single- and double Fano cavities is illustrated in figure \ref{fig:setup_sketch}. The specific part of the setup surrounding the cavity, outlined by the dashed line in figure \ref{fig:setup_sketch}, is subsequently shown in figure \ref{fig:setup_zoomed}. 

In order to effectively conduct the experiments in this project, it is imperative to be able to control certain parameters. Each element in the experimental setup is thoroughly considered for each their purpose in this regard, these will be outlined in this section. 

\begin{figure}[h!]
    \centering
    \includegraphics[width=0.6\textwidth]{figures/setup_sketch.pdf}
    \caption{Schematics of the experimental setup for measuring Fano cavity transmission- and Fano mirror transmission/reflection spectra. The diode laser \emph{DL} emits light into the setup through the optical fiber, the $\lambda/2$-waveplate \emph{HWP} and polarizing beam splitter \emph{PBS} ensures the light is linearly polarized and the optical telescope consisting of lenses $L_{1,2}$ modifies the beam waist to fit the given purpose. Detectors $P_{T,R,I}$ records the transmitted, reflected and incident light, respectively and the second HWP makes it possible to tune the polarization of the light just before the light is incident on the Fano cavity/mirror. The dashed line indicates the cavity setup seen in detail in figure \ref{fig:cavity_setup}. $I_{1-4}$ and $M_{1-3}$ indicate apertures/iris' and mirrors, respectively.}
    \label{fig:setup_sketch}
\end{figure}

\begin{figure}[h!]
    \centering
    \begin{subfigure}[b]{0.72\textwidth}
        \includegraphics[width=\textwidth]{figures/setup_skecth_zoomed.pdf}
        \caption{}
        \label{fig:setup_zoomed}
    \end{subfigure}
    \begin{subfigure}[b]{0.27\textwidth}
        \includegraphics[width=\textwidth]{figures/setup_cavity_picture.pdf}
        \caption{}
        \label{fig:cavity_setup_picture}
    \end{subfigure}
    \caption{(a) shows a sketch of the schematics of the Fano cavity setup located inside the dashed line in figure \ref{fig:setup_sketch}. The stages attached to both the top and bottom of the cavity are used to spacially align the Fano mirrors of the cavity, while the kinematics mirrors are used to adjust for the angular degrees of freedom. The piezo ring is used to scan for and tune the optimal length of the cavity and the rotational and adjustable xy-mounts are used for aligning the top of the cavity, when the bottom is aligned with respect to the laser, and both are hence fixed. (b) shows a picture of the cavity setup depicted schematically in (a). Note that the setup, both in (a) and (b), depicted is the one used to measure double Fano cavity transmission, and would thus be modified for a Fano mirror characterization.}
    \label{fig:cavity_setup}
\end{figure}

\subsubsection{Tunable diode laser}

As shown in figure \ref{fig:setup_sketch} the laser source used for the optical characterizations is coupled into the setup through an optical fiber. The laser used is a \emph{Toptica DLC Pro} tunable CW diode laser with a range for the transmission wavelength of $910$ $-$ $980nm$. The laser and controller are both depicted in figure \ref{fig:toptica_laser_and_controller}. The optical fiber is a \emph{P3-780PM-FC-10} fiber from Thorlabs which is a single mode\footnote{A single mode fiber can only sustain the TEM00 mode, which means that the output is known to be perfectly Gaussian.} polarization-maintaining optical fiber with an effective range of $770$ $-$ $1100nm$. Between the Toptica laser and the incoupling end of the fiber, a $\lambda/2$\emph{-plate} (HWP) and \emph{polarizing beam splitter} (PBS) is placed in order to be able to control the incident power of the laser and to only couple linearly polarized light into the setup.  

\begin{figure}[h!]
    \begin{subfigure}[b]{0.49\textwidth}
        \centering
        \includegraphics[height=4cm]{figures/toptica_laser.pdf}
        \caption{}
        \label{fig:toptica_laser}
    \end{subfigure}
    \begin{subfigure}[b]{0.49\textwidth}
        \centering
        \includegraphics[height=4cm]{figures/toptica_controller.pdf}
        \caption{}
        \label{fig:toptica_controller}
    \end{subfigure}
    \caption{The Toptica DLC Pro tunable CW diode laser (a) and the controller (b) used to tune the wavelength of the output beam.}
    \label{fig:toptica_laser_and_controller}
\end{figure}

The light being emitted from the optical fiber is sent through another HWP and PBS in order to be able to control the resulting polarization in the setup even more precisely, should there be any discrepancies of the light coupled into the fiber. The out-coupling fiber with the HWP and PBS is shown in figure \ref{fig:outcoupling_fiber}.

\begin{figure}[h!]
    \centering
    \includegraphics[width=0.5\textwidth]{figures/outcoupling_fiber.pdf}
    \caption{The out-coupling end of the P3-780PM-FC-10 polarization maintaining optical fiber from Thorlabs along with the HWP and PBS ensuring linear polarizaed light incident on the cavity or Fano mirror.}
    \label{fig:outcoupling_fiber}
\end{figure}

\subsubsection{$\lambda / 2$ - waveplate}

A $\lambda/2$-waveplate, or HWP, is constructed of a so-called bi-refringent material (most commonly crystaline quartz), which means that it has slightly different refractive indicies for incident light of different polarization axis'. Generally a HWP will have a \emph{fast}- and \emph{slow axis}, where it is understood that light polarized along the fast axis experiences a lower refractive index (and hence moves faster), than that along the slow axis. In this way the HWP separates the components of unpolarized light that has perpendicular and parallel polarizations with respect to the fast axis. 

\begin{figure}[h!]
    \centering
    \includegraphics[width=0.6\textwidth]{figures/HWP.pdf}
    \caption{A simple sketch of the effect of a HWP on linearly polarized light.}
    \label{fig:HWP}
\end{figure}

The effect of the HWP on linearly polarized light, is an effective rotation of the polarization, this is sketched simply in figure \ref{fig:HWP}. It can be shown that the polarization axis is rotated according to the angle between the fast axis of the HWP and the incident polarization axis. A relative angle $\theta$ will thus result in a rotation of $2\theta$, e.g. if $\theta=45^{\circ}$, this will constitute a rotation from completely p-polarized light to completely s-polarized light. This is the specific scenario sketched in figure \ref{fig:HWP}\cite{edmund_optics}. In this way a rotating HWP can allow one to alter an incident linearly polarized beam to be polarized along any axis, and is thus a necessary component for this particular setup.

\subsubsection{Optical telescope}

The linearly polarized light transmitted through the PBS passes through plano-convex lenses of positive focal lengths $f$ and $f^{\prime}$, $L_1$ and $L_2$, which together makes up an optical telescope used to manipulate the beam waist $w_0$ incident on the cavity or Fano mirror. 

Figure \ref{fig:telescope} shows schematics of the general way an optical telescope is utilized to manipulate the beam waist of a laser beam.

\begin{figure}[h!]
    \begin{subfigure}[b]{0.49\textwidth}
        \centering
        \includegraphics[height=4cm]{figures/optical_telescope.pdf}
        \caption{}
        \label{fig:telescope}
    \end{subfigure}
    \begin{subfigure}[b]{0.49\textwidth}
        \centering
        \includegraphics[height=4cm]{figures/optical_telescope_picture.pdf}
        \caption{}
        \label{fig:telescope_picture}
    \end{subfigure}
    \caption{(a) shows simple schematics of an optical telescope used to alter the waist of an incomming collimated beam, while (b) shows a picture taken of the actual optical telecope used in the setup.}
    \label{fig:telescope_sketch_and_pic}
\end{figure}

When the incident beam hits $L_1$ it is focused according to its focal length $f$, and by inserting another lens $L_2$ of a relatively longer focal length one can \emph{catch} the beam at the desired beam waist. If the focal length $f^{\prime}$ of $L_2$ is sufficently long, compared to the path of the beam after the lenses, the beam will be approximately collimated and remain at the waist obtained when incident on $L_2$.

\subsubsection{Transmission, reflection and incident photo-detectors}

After passing the optical telescope the beam reaches a simple 50/50 beam splitter (BS) which transmits 50\% of the light while reflecting another 50\%. The reflected light is then incident on the cavity or Fano mirror (target). 

The transmitted light passes through a lens $L_3$, which is focused on a photo-detector $P_I$ used for reference measurements and later normalization. Since the tunable laser in nature varies in power with the wavelength, it is necesarry to keep track of these fluctuations and correct for them in data analysis. 

The transmitted light is sent through, yet another, HWP which in this case is used only to alter the polarization of the light incident on the target. After the beam, or a portion of it, has passed the target it is sent through a lens $L_5$ focused onto the transmission detector $P_T$.

The part of the light incident on the target that is \emph{not} transmitted, is reflected back onto the BS which then transmits 50\% once again, and thus reflects the other 50\%. The transmitted part here is focused through the lens $L_4$ onto the reflection detector $P_R$.

\subsubsection{The double Fano cavity measurement setup}

The cavity measurement setup shown in figure \ref{fig:setup_zoomed} is the one used to measure the transmission spectrum of a double Fano cavity consisting of two Fano mirrors. This part of the setup consists of one set of standard \emph{PT1} $\mu m$-stages from Thorlabs combined with an additional set of \emph{XRN25/M} stages, also from Thorlabs, to provide precise movement of each Fano mirror in the xy-plane. 

\begin{figure}[h!]
    \centering
    \begin{subfigure}[b]{0.6\textwidth}
        \includegraphics[width=\textwidth]{figures/setup_bottom.pdf}
        \caption{}
        \label{fig:setup_bottom}
    \end{subfigure}
    \begin{subfigure}[b]{0.3\textwidth}
        \includegraphics[width=\textwidth]{figures/cavity_setup_bottom_pic.pdf}
        \caption{}
        \label{fig:setup_bottom_pic}
    \end{subfigure}
    \caption{(b) shows a picture taken of the \emph{bottom} part of the double Fano cavity setup, while (a) depicts the corresponding schematics.}
    \label{fig:setup_bottom_sketch_and_pic}
\end{figure}

Examining the structure from the bottom (as it is built), a kinematic mirror mount is attached to the lower set of xy-stages, this is used to control the angular degrees of freedom of the Fano mirror which makes up the bottom of the cavity. On the mirror mount, a \emph{NAC2123} piezo ring actuator from Noliac is firmly attached and connected to a piezo driver. The driver is capable of applying a fixed current, and thus manually controlling the piezo expansion, but is also connected to a frequency generator. The signal from the frequency generator can be used to modulate the piezo expansion by an alternating current, which practically scans the cavity length in a range according to the effective free stroke of the piezo ring\footnote{The NAC2123 piezo ring has an outer/inner diameter of $12mm/6mm$ and a height of $2mm$. Its listed free stroke is $3.3 \mu m$, which practically is likely slightly limited in the configuration used. To obtain the precise free stroke one can simply count the number of fringes resolved during one period as the length-dependent FSR is known to be $\lambda/2$.}. Lastly, in order to be able to place the Fano mirror on the  piezo ring, a ceramic Macor membrane holder is used. The bottom part of the cavity setup is highlighted in figure \ref{fig:setup_bottom}.

The part of the setup built to control the Fano mirror making up the top of the cavity is slightly more complicated, as this is the Fano mirror that is, for practical reasons, aligned last. This means that additional degrees of freedom must be controled by movement of the Fano mirror itself. The alignment procedure will be explained in detail in sections \ref{sec:grating_characterization} and \ref{sec:cavity_measurements}.

\begin{figure}[h!]
    \centering
    \begin{subfigure}[b]{0.6\textwidth}
        \includegraphics[width=\textwidth]{figures/setup_top.pdf}
        \caption{}
        \label{fig:setup_top}
    \end{subfigure}
    \begin{subfigure}[b]{0.3\textwidth}
        \includegraphics[width=\textwidth]{figures/setup_top_pic.pdf}
        \caption{}
        \label{fig:setup_top_pic}
    \end{subfigure}
    \caption{(b) shows a picture taken of the \emph{top} part of the double Fano cavity setup, while (a) depicts the corresponding schematics.}
    \label{fig:setup_top_sketch_and_pic}
\end{figure} 

The top part of the cavity setup is attached to the second set of xy-stages and additionally to an \emph{NFL5DP20} stage from Thorlabs in the z-direction in order to be able to change the length of the cavity. As for the bottom part of the cavity setup, a kinematic mirror mount acts as the base of the construction. This is, once again, to control the angular degrees of freedom of the corresponding Fano mirror. This mirror mount is equipped with piezo elements to control the angular degrees of freedom, by applied voltage, with higher resolution than for the manual adjustments. On the mirror mount, a standard rotational mount with a 1 inch inner winding is attached in order to be able to control the rotational degree of freedom of the Fano mirror. An additional xy-adjustable mount is then used in order to effectively place the Fano mirror in the center of the rotational mount to ensure the rotational axis is in the center of the membrane. Finally, the Fano mirror is taped to a costume mount created to fit into the xy-adjustable mount. The top part of the cavity setup is highlighted in figure \ref{fig:setup_top}, and presented separate from the setup to show how the Fano mirror is attached in figure \ref{fig:cavity_setup_top_separate_pic}.

\begin{figure}[h!]
    \centering
    \includegraphics[width=0.4\textwidth]{figures/cavity_setup_top_separate_pic.pdf}
    \caption{The top part of the cavity setup separated from the rest and flipped. Note that the Fano mirror attached with the tape is shown. The three wires visible are then ones that connect the piezo elemehts with a driver.}
    \label{fig:cavity_setup_top_separate_pic}    
\end{figure}

What has been outlined here is the setup utilized to optically characterize the double Fano cavity. If one wishes to do so for the single Fano cavity instead, the setup would be modified such that the top part of the setup, highlighted in figure \ref{fig:setup_top}, would only consist of the kinematic mirror mount. Inside the mirror mount would then be placed a broadband mirror. The rotational- and xy-adjustable mounts would here be redundant due to the rotational symmetry of any standard broadband mirror.

\subsubsection{Vibrational noise reduction}\label{sec:vibrational_noise_reduction}

When the cavity measurements were conducted it was apparent that the double Fano cavity was particularly prone to noise. While noise is not a very precise definition in its own, it did seem that the most prominent source was associated with the length of the cavity $l$. When applying a constant voltage to the piezo element depicted in figure \ref{fig:setup_bottom_sketch_and_pic}, in order to achieve the correct length for sustaining the Fano resonance, the signal started to fluctuate dramatically when approaching the resonance. 

\begin{figure}[h!]
    \centering
    \begin{subfigure}[b]{0.49\textwidth}
        \centering
        \includegraphics[height=4.5cm]{figures/noise_box_front.pdf}
        \caption{}
        \label{fig:box_front}
    \end{subfigure}
    \begin{subfigure}[b]{0.49\textwidth}
        \centering
        \includegraphics[height=4.5cm]{figures/noise_box_side.pdf}
        \caption{}
        \label{fig:box_side}
    \end{subfigure}
    \caption{Front- (a) and side (b) views of the plexi-glass box used to reduce the acoustic noise in the setup.}
    \label{fig:noise_box}
\end{figure}

By examining the characteristics of the noise when apparent, the source appeared to be acoustic in origin, i.e. making a sudden sound in close vicinity of the setup caused a spike in the noise level.

In order to reduce the acoustic noise, a plexi-glas box was placed around the entire setup as seen in figure \ref{fig:noise_box}. This proved to reduce the noise, and this improve the signal-to-noise ratio substantially.

While the signal-to-noise ratio was improved, the noise was still there, and thus additional measures in order to reduce this was taken. Two slabs of \emph{teflon} were introduced into the setup, one between the optical table and the first xy-stage in the cavity setup, and the next between the first and second xy-stage. The reasononing behind this imlpementation relied on the difference in acoustic impedence of teflon relative to aluminum. A high change in acoustic impedence creates an interface which should reflect any unwanted vibrations propagating through the optical table. This further reduced the noise, while it regretably did not remove it completely. 

\subsubsection{Additional equipment used}

In order to record a measurement of any kind in the setup, a \emph{Keysight InfiniiVision DSOX2024A} oscilloscope was connected with all the detectors $P_T$, $P_R$ and $P_I$ in the setup. The oscilloscope, along with the Toptica laser, was then controlled by a Matlab script implemented by previous students/researchers of the lab. 

Scanning the piezo element, by application of an altertnating current also required additional equipment, and more specifically a frequency generator capable of generating a triangular signal with an offset. The frequency generator used was a \emph{Keysight 33500B Waveform Generator}. Insight on scanning the cavity length using the piezo will be provided in section \ref{sec:cavity_measurements}. The oscilloscope and frequency generater are shown in figure \ref{fig:scope_and_freq_generator}.

\begin{figure}[h!]
    \centering
    \begin{subfigure}[b]{0.49\textwidth}
        \centering
        \includegraphics[height=4cm]{figures/scope.pdf}
        \caption{}
        \label{fig:scope}
    \end{subfigure}
    \begin{subfigure}[b]{0.49\textwidth}
        \centering
        \includegraphics[height=4cm]{figures/frequency_generator.pdf}
        \caption{}
        \label{fig:freq_generator}
    \end{subfigure}
    \caption{Pictures of the oscilloscope (a) and frequency generator (b) used during experimental investigations in this project.}
    \label{fig:scope_and_freq_generator}
\end{figure}
