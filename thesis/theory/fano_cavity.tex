\section{The Fano mirror: a sub wavelength grating}

\subsection{Geometric optical analysis}

Considering an ideal grating with period \emph{d} in the sub-wavelength regime, it can be shown that only a single mode of reflection/transmission is supported. 

Any grating, of arbitrary dimensions, must comply with the very general \emph{grating equation}\cite{Pedrotti} given as
\begin{equation}
    \sin \theta_m = m \frac{\lambda}{d},
\end{equation}
for the special case of a linearly polarized plane wave incident on a grating placed normal to the direction of propagation. Now, inserting the sub-wavelength condition \emph{d << $\lambda$}, it is easily seen that the right side of the equation blows up for any order of reflection $|m| > 0$, effectively showing that this is the aforementioned single supported mode in this regime. Furthermore, it can be equally easily seen that the propagation direction of the 0'th order mode is the same as the incident beam, i.e. normal to the grating.

\subsection{Reflection/transmission spectra and line shape analysis}

\subsubsection{Lossless grating}

We wish to analytically describe the wavelength-dependent spectra for the transmission and reflectivity of an infinite sub-wavelength grating. By first considering the case where absorption and thermal coupling effects are neglected, i.e. a lossless grating, we can assume conservation of energy and thereby the relations
\begin{equation}
    |r_g|^2+|t_g|^2=1 \hspace{0.5cm} \text{and} \hspace{0.5cm} |r_d|^2+|t_d|^2=1,
    \label{eq:energy conservation}
\end{equation}
where the subscripts \emph{g} and \emph{d} indicate the \emph{grating} and \emph{direct} transmissions and reflectivities, respectively. It is implied that the direct coefficients are constants and describe the transmission and reflectivity when the incident wavelength is significantly detuned from any guided-mode resonance of the grating. Furthermore, it is also implied that the grating coefficients are functions of the incident wavelength.

We now assume a normal incident beam on the grating as a linearly polarized monochromatic plane wave, with a wavelength close to a guided-mode resonance of the grating. In order to describe the coefficients $r_g$ and $t_g$ we follow the formalism presented by Fan and Joannopoulos \cite{Fan-Joannopoulos-guided-mode-resonance} and consider the likely paths of the incident light through the grating. It is quite intuitive to consider the case where the light is simply transmitted, and this shall be our first case hereafter denoted the \emph{direct pathway}. Another case one might consider is the one where the incident light excites the guided-mode resonance in the grating, thus causing interference. This case is denoted the \emph{indirect pathway} and decays more slowly than it's direct counterpart. 

The interference caused when the guided mode is excited is often referred to as \emph{Fano resonances}, due to its physical similarities to the description of interference between a discrete autoionized state and a bound continuum state first reported by Fano \cite{Fano-theory}. The cross section of inelastic scattering, when measured as a function of energy, showed characteristic asymmetric peaks. These were described as the aforementioned interference pattern between \emph{direct} (the discrete state) and \emph{indirect} (the continuum state) pathways. 

By generalizing the model of Fan and Joannopoulos \cite{Fan-Joannopoulos-guided-mode-resonance} we describe the transmission and recletivity coefficent amplitudes as 
\begin{equation}
    r_g = r_d + \frac{a}{k-k_1 + i\gamma} \hspace{0.5cm} \text{and} \hspace{0.5cm} t_g = t_d + \frac{b}{k-k_1+i\gamma},
    \label{eq:ref/trans}
\end{equation}
where $k=2\pi/\lambda$ is the incident wave number, $k_1 = 2\pi/\lambda_1$ is the wave number according to the guided-mode resonance and $\gamma$ is the HWHM (half width at half maximum) of the guided-mode resonance. Complex coefficients $a$ and $b$ describe the interference between the directly transmitted or reflected waves and the guided mode of the grating. 

Note that in eq. (\ref{eq:ref/trans}) the right side of the expression for each coefficient corresponds to the continuum state i.e. the indirect pathway, while the direct transmission and reflection coefficients take the role of the autoionized discrete state, i.e. the direct pathway\footnote{The general eigenvector of a state comprised of a super-position between a discrete state and a continuum, i.e. a state vector corresponding to a Fano resonance, is given as $\Psi_E = a\phi + \int dE^{\prime} b_{E^{\prime}} \psi_{E^{\prime}}$, given in eq. (2) in ref. \cite{Fano-theory}, where $a$ and $b_{E^{\prime}}$ describes the probability of either pathway.} \cite{Fano-theory}

As we are dealing with an ideal, lossless, grating, we assume coefficients $a$ and $b$ to be equal, meaning that we specifically assume vertical symmetry throughout the grating. By considering eq. (\ref{eq:energy conservation}) this in turn leads to 
\begin{equation}
    a = b = -i \gamma (t_d + r_d),
\end{equation}
which further yields an expression for the grating transmission amplitude coefficient on the form
\begin{equation}
    t_g = t_d \frac{k - k_0}{k - k_1 + i \gamma}.
    \label{eq:lossless transmission coefficient}
\end{equation}
Here, the newly introduced $k_0 = 2\pi/\lambda_0$ is the zero-transmission/unity-reflectivity wave number.

To generalize eq. (\ref{eq:lossless transmission coefficient}) to include non-unity reflectivity and non-zero transmission, we allow for $a \neq b$ meaning that the case of vertical asymmetry is included in the model. By assuming $r_d,t_d \in \mathbb{R}$, eq. (\ref{eq:energy conservation}) leads to the coupled differential equations
\begin{equation}
    \begin{split}
        &t_d x_a + r_d x_b = 0, \hspace{0.3cm} \text{and} \\
        &x_a^2 + y_a^2 + x_b^2 + y_b^2 + 2 t_d \gamma y_a + 2 r_d \gamma y_b = 0,
    \end{split}
    \label{eq:lossless couples diff. eqs.}
\end{equation}
where $\{x,y\}_{a,b}$ respectively denotes the real and imaginary parts of the coefficients $a$ and $b$. Solving eqs. (\ref{eq:lossless couples diff. eqs.}) leads to the correct complex reflectivity coefficients and the expression for the transmission coefficient amplitudes now reads
\begin{equation}
    t_g = t_d \frac{k - k_0 + i \beta}{k - k_1 + i \gamma},
    \label{eq:lossy transmission coefficients}
\end{equation}
where $k_0$ and $\beta$ are defined from the expression for $a$ found by solving eqs. (\ref{eq:lossless couples diff. eqs.}), given as
\begin{equation}
    a = t_d (k_1 - k_0 - i \gamma + i \beta).
\end{equation}
Finally, this allows for non-zero transmission and non-unity reflectivity at wave number $k_0$.

\subsubsection{Lossy grating}

In order to modify the above model such that losses, e.g. due to absorption or thermal coupling effects, are accounted for, we add a resonant loss term to the energy conservation relation in eq. (\ref{eq:energy conservation}). For this we introduce the resonant loss level $L$, which must be known in order to accurately calculate the complex reflectivity coefficients. The energy conservation relation is modified such that
\begin{equation}
    |t_g|^2 + |r_g|^2 + \frac{c^2}{(k - k_1)^2 + \gamma^2} = 1,
    \label{eq:lossy energy conservation}
\end{equation}
where the coefficient $c^2 = L((k-k_1)^2 + \gamma^2)$ includes the resonant loss term $L$. A new set of coupled differential equations are found, using eq. (\ref{eq:lossy energy conservation}), given as
\begin{equation}
    \begin{split}
        &t_d x_a + r_d x_b = 0, \hspace{0.3cm} \text{and} \\
        &x_a^2 + y_a^2 + x_b^2 + y_b^2 + c^2 +  2 t_d \gamma y_a + 2 r_d \gamma y_b = 0.
    \end{split}
    \label{eq:lossy couples diff. eqs.}
\end{equation}
It is easily identified that eq. (\ref{eq:lossless couples diff. eqs.}) and eq. (\ref{eq:lossy couples diff. eqs.}) differ only by the addition of coefficient $c^2$, and thereby the losses. Solving eq. (\ref{eq:lossy couples diff. eqs.}) leads to the correct complex reflectivity coefficients, except that they now account for any losses associated with the grating. 

In conclusion, the complete grating model consists of an expression for the transmission coefficients and a set of coupled differential equations for the reflection coefficients, shown in eq. (\ref{eq:lossy transmission coefficients}) and eq. (\ref{eq:lossy couples diff. eqs.}), respectively. The model on the form used for this project and subsequent thesis is derived in previous work by A. Darki et al. \cite{Darki} and more recently T. Mitra et al. \cite{Mitra}.

\section{The single Fano cavity}

\section{The double Fano cavity}

\subsection{Symmetric}

\subsection{Asymmetric}