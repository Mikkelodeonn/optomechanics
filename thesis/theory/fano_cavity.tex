\clearpage
\subsection{The Fano mirror: a sub wavelength grating}\label{sec:fano_mirror}

\subsubsection{Reflection/transmission spectra and line shape analysis}

\subsubsection{Lossless grating}

We wish to analytically describe the wavelength-dependent spectra for the transmission and reflectivity of an infinite sub-wavelength grating. The wavelength dependence arises from the fact that any incident light will be subject to interactions with the so-called \emph{guided-mode} of the grating. The grating is, in this way, said to act as an effective $\text{TM}_0$ waveguide, ensuring resonant modes will only be reflected in the 0'th order. 

By first considering the case where absorption and thermal coupling effects are neglected, i.e. a lossless grating, we can assume conservation of energy and thereby the relations
\begin{equation}
    |r_g|^2+|t_g|^2=1 \hspace{0.5cm} \text{and} \hspace{0.5cm} |r_d|^2+|t_d|^2=1,
    \label{eq:energy conservation}
\end{equation}
where the subscripts \emph{g} and \emph{d} indicate the \emph{grating} and \emph{direct} transmissions and reflectivities, respectively. It is implied that the direct coefficients are constants and describe the transmission and reflectivity when the incident wavelength is significantly detuned from any guided-mode resonance of the grating. Furthermore, it is also implied that the grating coefficients are functions of the incident wavelength.

We now assume a normal incident beam on the grating as a linearly polarized monochromatic plane wave, with a wavelength close to a guided-mode resonance of the grating. In order to describe the coefficients $r_g$ and $t_g$ we follow the formalism presented by Fan and Joannopoulos \cite{Fan-Joannopoulos-guided-mode-resonance} and consider the likely paths of the incident light through the grating. It is quite intuitive to consider the case where the light is simply transmitted, and this shall be our first case hereafter denoted the \emph{direct pathway}. Another case one might consider is the one where the incident light excites the guided-mode resonance in the grating. This case is denoted the \emph{indirect pathway} and decays more slowly than it's direct counterpart. 

The interference caused when the guided mode is excited is often referred to as \emph{Fano resonances}, due to its physical similarities to the description of interference between a discrete autoionized state and a bound continuum state first reported by Fano \cite{Fano-theory}. The cross section of inelastic scattering, when measured as a function of energy, showed characteristic asymmetric peaks. These were described as the aforementioned interference pattern between \emph{direct} (the discrete state) and \emph{indirect} (the continuum state) pathways. 

By generalizing the model of Fan and Joannopoulos \cite{Fan-Joannopoulos-guided-mode-resonance} we describe the transmission and reflectivity coefficent amplitudes as 
\begin{equation}
    r_g = r_d + \frac{a}{k-k_1 + i\gamma} \hspace{0.5cm} \text{and} \hspace{0.5cm} t_g = t_d + \frac{b}{k-k_1+i\gamma},
    \label{eq:ref/trans}
\end{equation}
where $k=2\pi/\lambda$ is the incident wave number, $k_1 = 2\pi/\lambda_1$ is the wave number according to the guided-mode resonance and $\gamma$ is the HWHM (half width at half maximum) of the guided-mode resonance. Complex coefficients $a$ and $b$ describe the interference between the directly transmitted or reflected waves and the guided mode of the grating. 

Note that in eq. (\ref{eq:ref/trans}) the right side of the expression for each coefficient corresponds to the continuum state i.e. the indirect pathway, while the direct transmission and reflection coefficients take the role of the autoionized discrete state, i.e. the direct pathway\cite{Fano-theory}.

As we are dealing with an ideal, lossless, grating, we assume coefficients $a$ and $b$ to be equal, meaning that we specifically assume vertical symmetry throughout the grating. By considering eq. (\ref{eq:energy conservation}) this in turn leads to 
\begin{equation}
    a = b = -i \gamma (t_d + r_d),
\end{equation}
which further yields an expression for the grating transmission amplitude coefficient on the form
\begin{equation}
    t_g = t_d \frac{k - k_0}{k - k_1 + i \gamma}.
    \label{eq:lossless transmission coefficient}
\end{equation}
Here, the newly introduced $k_0 = 2\pi/\lambda_0 = k_1 -i \gamma (r_d/t_d)$ is the zero-transmission/unity-reflectivity wave number.

To generalize eq. (\ref{eq:lossless transmission coefficient}) to include non-unity reflectivity and non-zero transmission, we allow for $a \neq b$\cite{Bykov}\cite{Darki2}\cite{Parthenopoulos}, meaning that the case of vertical asymmetry is included in the model\cite{Popov}. By assuming $r_d,t_d \in \mathbb{R}$, eq. (\ref{eq:energy conservation}) leads to the coupled differential equations
\begin{equation}
    \begin{split}
        &t_d x_a + r_d x_b = 0, \hspace{0.3cm} \text{and} \\
        &x_a^2 + y_a^2 + x_b^2 + y_b^2 + 2 t_d \gamma y_a + 2 r_d \gamma y_b = 0,
    \end{split}
    \label{eq:lossless couples diff. eqs.}
\end{equation}
where $\{x,y\}_{a,b}$ respectively denotes the real and imaginary parts of the coefficients $a$ and $b$. Solving eqs. (\ref{eq:lossless couples diff. eqs.}) leads to the correct complex reflectivity coefficients and the expression for the transmission coefficient amplitudes now reads
\begin{equation}
    t_g = t_d \frac{k - k_0 + i \beta}{k - k_1 + i \gamma},
    \label{eq:transmission_coefficients_non_zero_values}
\end{equation}
where $k_0$ and $\beta$ are defined from the expression for $a$ found by solving eqs. (\ref{eq:lossless couples diff. eqs.}), given as
\begin{equation}
    a = t_d (k_1 - k_0 - i \gamma + i \beta).
\end{equation}
Finally, this allows for non-zero transmission and non-unity reflectivity at wave number $k_0$.

To show the validity of the model at this point, we introduce a periodic grating which is arbitrarily sketched in figure \ref{fig:MIST_grating_sketch}. The sketch indicates the period of the grating $\Lambda$, the top finger width $w_t$, the offset between the finger top and the bottom of the grating $x$, the total grating thickness $t$ and finger depth $d$. Furthermore, the grating is patterned in a \emph{silicon nitride (SiN)} membrane  of refractive index $n_{SiN}$. 

\begin{figure}[h!]
    \centering
    \includegraphics[width=0.7\textwidth]{figures/grating_MIST_sketch.pdf}
    \caption{Schematics of a periodic optical $SiN$ grating with phyiscal parameters: $\Lambda = 857$nm, $w_t = 623$nm, $h = 55.1$nm, $t = 152$nm, $d = 67$nm, $n_{SiN} = 2.15$.}
    \label{fig:MIST_grating_sketch}
\end{figure}


In order to simulate the transmission and reflectivity profiles as a function of the incident wavelength, we utilize the \emph{Modeled Integrated Scatter Tool (MIST)} developed by the \emph{National Institute of Standards and Technology (NIST)}\cite{MIST}. The simulation solves Maxwell's equations for any pre-defined infinite periodic structure. 

MIST assumes an incident plane-wave, and hence predicts the \emph{ideal} spectra for the transmission and reflectivity, i.e. zero and unity when on resonance, respectively. In order to include effects related to the Gaussian behaviour of a more realistic incident beam, such as collimation and finite-size effects\cite{Toft-Vandborg}, one would have to solve Maxwell's equations for a Gaussian distribution. We safely assume an incident plane-wave for reasons related to the order of magnitude for the \emph{Rayleigh range} of the beam used, compared with that of the spacial range of the conducted experiments.

In reality, the interference inside the cavity however cannot be perfect, as any Gaussian beam can be represented by a number rays, infinitesimal in size, which would all track differently through the grating, resulting in non-unity reflectivity and non-zero transmission. In order to synthetically correct for this we scale each transmission value found by MIST according to
\begin{equation}
    t_g = (1 - \varepsilon) \cdot t_{MIST} + \varepsilon,
    \label{eq:trans_correction_MIST}
\end{equation}
where we define the correction to be arbitrarily small as $\varepsilon = 1\%$.

\begin{figure}[h!]
    \centering
    \includegraphics[width=0.7\textwidth]{figures/MIST_grating_sim.pdf}
    \caption{Reflectivity and transmission values (scaled and unscaled) found from a MIST simulation plotted with the corresponding least squares fit to the model in eq. (\ref{eq:transmission_coefficients_non_zero_values}). The resulting grating parameters are given as $\lambda_0 = 951.208$nm, $\lambda_1 = 951.356$nm, $t_d = 80.94\%$, $r_d = 52.7\%$, $\gamma_{\lambda} = 0.527$, and $\beta = 4.42 \cdot 10^{-7}$.}
    \label{fig:MIST_sim_of_grating}
\end{figure}

Figure \ref{fig:MIST_sim_of_grating} shows the scaled and unscaled results of the simulation using MIST for the following physical grating parameters:
\begin{equation}
    \begin{split}
    \Lambda = &857 \text{nm},\:\: w_t = 623 \text{nm},\:\: h = 55.1 \text{nm},\\ &t = 152 \text{nm},\:\: d = 67 \text{nm}\: \text{ and }\: n_{SiN} = 2.15,
    \end{split}
    \label{eq:physical_grating_params}
\end{equation}

The reflectivity- and (corrected) transmission values were then fitted to the model in eq. (\ref{eq:transmission_coefficients_non_zero_values}) using a least squares fitting method, and plotted along with the simulated values. The resulting grating parameters were found as
\begin{equation}
    \begin{split}
    \lambda_0 = &951.208 \text{nm},\:\: \lambda_1 = 951.356 \text{nm},\:\: t_d = 80.94\%,\\ &r_d = 52.7\%,\:\: \gamma_{\lambda} = 0.527 \text{nm}\: \text{ and }\: \beta = 4.42 \cdot 10^{-7},
    \end{split}
    \label{eq:optical_grating_params}
\end{equation}
where $\lambda_0$ is the cavity mode resonance wavelength, $\lambda_1$ is the guided-mode resonance wavelength, $r_d$ ($t_d$) is the direct reflectivity (transmission), $\gamma_{\lambda}$ is the width of the guided-mode resonance and $\beta$ is a constant associated with non-unity reflectivity and non-zero transmission. 

\subsubsection{Lossy grating}

In order to modify the above model such that losses, e.g. due to absorption or thermal coupling effects, are accounted for, we add a resonant loss term to the energy conservation relation in eq. (\ref{eq:energy conservation}). For this we introduce the resonant loss level $L$, which must be known in order to accurately calculate the complex reflectivity coefficients. The energy conservation relation is modified such that
\begin{equation}
    |t_g|^2 + |r_g|^2 + \frac{c^2}{(k - k_1)^2 + \gamma^2} = 1,
    \label{eq:lossy energy conservation}
\end{equation}
where the coefficient $c^2 = L((k-k_1)^2 + \gamma^2)$ includes the resonant loss term $L$. A new set of coupled differential equations are found, using eq. (\ref{eq:lossy energy conservation}), given as
\begin{equation}
    \begin{split}
        &t_d x_a + r_d x_b = 0, \hspace{0.3cm} \text{and} \\
        &x_a^2 + y_a^2 + x_b^2 + y_b^2 + c^2 +  2 t_d \gamma y_a + 2 r_d \gamma y_b = 0.
    \end{split}
    \label{eq:lossy couples diff. eqs.}
\end{equation}
It is easily identified that eq. (\ref{eq:lossless couples diff. eqs.}) and eq. (\ref{eq:lossy couples diff. eqs.}) differ only by the addition of coefficient $c^2$, and thereby the losses. Solving eq. (\ref{eq:lossy couples diff. eqs.}) leads to the correct complex reflectivity coefficients, except that they now account for any losses associated with the grating. 

In conclusion, the complete grating model consists of an expression for the transmission coefficients and a set of coupled differential equations for the reflection coefficients, shown in eq. (\ref{eq:transmission_coefficients_non_zero_values}) and eq. (\ref{eq:lossy couples diff. eqs.}), respectively. The model on the form used for this project and subsequent thesis is derived in previous work by A. Darki et al. \cite{Darki} and more recently T. Mitra et al. \cite{Mitra}.

Figure \ref{fig:lossy_grating_spectrum} shows reflection and transmission spectra of a grating of parameters given in eq. \ref{eq:optical_grating_params} with a synthetic non-zero resonance loss term in order to show the effect of including losses. It is seen from the added \emph{loss curve} that the losses increase when approaching the resonance wavelength, as the interaction with the guided-mode, and thus the grating, gets stronger in this region. 
\begin{figure}[h!]
    \centering
    \includegraphics[width=0.7\textwidth]{figures/arbitrary_lossy_grating_ref_trans.pdf}
    \caption{Simulated reflectiviy and transmission spectra of a grating of parameters given in eq. (\ref{eq:optical_grating_params}) with a synthetically added non-zero loss term, and the corresponding grating losses.}
    \label{fig:lossy_grating_spectrum}
\end{figure}

\subsection{The single Fano cavity}

\subsubsection{The single Fano cavity model}

The single Fano cavity consists of a planar broadband mirror, and a sub-wavelength grating, i.e. a Fano mirror, as described in section \ref{sec:fano_mirror} and seen in figure \ref{fig:broadband_and_single_fano_sketch} where schematics of the single Fano and broadband cavity configurations are shown. While the broadband mirror has fixed optical properties, the Fano mirror has transmission and reflection coefficients dependent on the incident wavelength, according to solutions to the coupled differential equations of eq. (\ref{eq:lossy couples diff. eqs.}). 

\begin{figure}[h!]
    \centering
    \begin{subfigure}[b]{0.3\textwidth}
        \includegraphics[width=\textwidth]{figures/broadband_sketch.pdf}
        \caption{}
        \label{fig:broadband_sketch}
    \end{subfigure}
    \hspace{1cm}
    \begin{subfigure}[b]{0.3\textwidth}
        \includegraphics[width=\textwidth]{figures/single_fano_sketch.pdf}
        \caption{}
        \label{fig:single_fano_sketch}
    \end{subfigure}
    \caption{In (a) is seen the schematics of a cavity consisting of two broadband mirrors with transmission and reflectivity coefficients $t_m$, $r_m$, $t_m^{\prime}$, and $r_m^{\prime}$. (b) shows the schematics of a single Fano cavity consisting of one broadband mirror with coefficients $t_m$, $r_m$ and one Fano mirror with wavelength dependent coefficients $t_g$, $r_g$. The reflectors in either cavity are separated by a cavity length $l$.}
    \label{fig:broadband_and_single_fano_sketch}
\end{figure}

In order to model the single Fano cavity transmission spectra, we therefore consider the transmission function of a normal incident and planer Fabry-Perot cavity in eq. (\ref{eq:fabry_perot_trans}) with $r,t \rightarrow r_m,t_m$ and $r^{\prime},t^{\prime} \rightarrow r_g(\lambda),t_g(\lambda)$\cite{Naesby}. Here the subscript $m$ indicates the broadband \emph{mirror} coefficients, and $g$ is for \emph{grating}, which indicates the coefficients of the Fano mirror. Rewriting eq. (\ref{eq:fabry_perot_trans}) such that it describes the normalized transmission amplitude, through the single Fano cavity, $T_{cav} = |E_{out}|^2/|E_{0,in}|^2$ we get
\begin{equation}
    T_{cav} = \left|\frac{t_m t_g(\lambda) e^{i\phi}}{1 - r_m r_g e^{2i\phi}}\right|^2,
    \label{eq:single_fano_trans}
\end{equation}
where $\phi = 2\delta = kl$, $k=2 \pi / \lambda$ and $l$ is the cavity length, as is consistent the general case described in section \ref{sec:fano_mirror}.

\subsubsection{Transmission linewidth}\label{sec:single_fano_cavity_trans_linewidth}

In order to analytically describe how the transmission spectrum at, or close to, the overall resonance\footnote{This terminology is used for the scenario where the guided-mode, cavity mode and input laser mode all meet the resonance condition $\lambda_g \approx \lambda_c \approx \lambda_l$, where $g,c,l$ stands for grating/guided-mode, cavity and laser, respectively.} behaves as a function of the incident wavelength, we attempt to generalize the Fano model for this specific scenario. Considering the case where the cavity resonance closely resembles the guided-mode resonance of the Fano mirror (the zero-transmission wavelength), eq. (\ref{eq:single_fano_trans}) can be approximated well by
\begin{equation}
    T_{cav} \approx \frac{A}{1 + \left( \frac{\Delta}{1 - \nu \Delta} \right)^2},
    \label{eq:general_fano_model}
\end{equation}
where $\Delta = (\lambda - \lambda_c) / \delta \lambda$ is the detuning from the cavity resonance normalized by the HWHM $\delta \lambda$, and $\nu$ is a constant describing the asymmetry of the single Fano transmission spectrum. \cite{Mitra}\cite{Darki} 

From eq. (\ref{eq:general_fano_model}) it can be shown that the HWHM of the Fano transmission profile around the overall resonance wavelength, i.e. when $\lambda_c \approx \lambda_0$, is approximately given as
\begin{equation}
    \delta \lambda \approx \frac{1}{\frac{1}{\delta \lambda_c} + \frac{1}{\delta \lambda_g}},
    \label{eq:analytical_linewidth}
\end{equation}
where 
\begin{equation}
    \delta \lambda_c = \frac{\lambda_0^2}{8 \pi l} (|t_g(\lambda_0)|^2 + |t_m|^2 + L)
    \label{eq:analytical_linewidth_broadband}
\end{equation}
is the HWHM of a broadband cavity and
\begin{equation}
    \delta \lambda_g = \frac{\gamma_{\lambda}}{2 (1-r_d)}(|t_g(\lambda_0)|^2 + |t_m|^2 + L)
    \label{eq:analytical_linewidth_single_fano}
\end{equation}
is the HWHM of the Fano cavity in the so-called Fano regime.\cite{Mitra}\cite{Darki} In eqs. (\ref{eq:analytical_linewidth})-(\ref{eq:analytical_linewidth_single_fano}) $\lambda_0$ is the Fano cavity resonance wavelength, $l$ is the cavity length, $L = \left(1 - |r_g(\lambda_0)|^2 - |t_g(\lambda_0)|^2\right)$ is the total additional losses of the cavity when on resonance, $\gamma \lambda$ is the width of the guided-mode resonance of the Fano mirror and $r_d$ is the off-resonance, or \emph{direct}, reflectivity of the Fano mirror.

The \emph{Fano regime} and its counterpart the so-called \emph{standard regime} are defined for a given single Fano cavity, by its length $l$. By inspection of eqs. (\ref{eq:analytical_linewidth_broadband}) and (\ref{eq:analytical_linewidth_single_fano}) it is seen that for $l \rightarrow \infty$ the linewidth in eq. (\ref{eq:analytical_linewidth}) is dominated by the broadband cavity term, while for the opposite case, $l \rightarrow 0$, the linewidth is predominantly given by the Fano cavity term. 

Generally the Fano regime describes the cavity lengths for which eq. (\ref{eq:analytical_linewidth}) shows a significant divergence from the broadband linewidth in eq. (\ref{eq:analytical_linewidth_broadband}). Oppositely, when in the standard regime the broadband and Fano cavity produces resonance transmission peaks of comparable, if not equal, linewidths.

Figures \ref{fig:standard_regime_trans} and \ref{fig:fano_regime_trans} depicts two examples, in each their regime, of single Fano cavity transmission spectra together with their respective complimentary broadband cavity transmission profiles, and the reflectiviy amplitude of the Fano mirror used to model both profiles. As shown in section \ref{sec:fano_mirror} any Fano mirror, or grating, is described by five characterizing parameters. For the one used to model the transmission spectra in figures \ref{fig:standard_regime_trans} and \ref{fig:fano_regime_trans}  these are 
\begin{equation}
    \begin{split}
    \lambda_0 = &951 nm,\:\: \lambda_1 = 951 nm,\:\: t_d = 80\%,\\ &\gamma \lambda = 0.5 nm\: \text{ and }\: \beta = 0,
    \end{split}
\end{equation}
where we remember that $\lambda_{0,1}$ are the resonant wavelengths for, respectively, the cavity- and guided-modes, $t_d$ is the direct transmission, $\gamma \lambda$ is the width of the guided-mode resonance and $\beta$ is a constant describing the losses of the grating, which here are set to $0$, and thus neglected. The broadband mirror parameters are chosen such that they perfectly resemble the transmission and reflectivity of the Fano mirror on resonance, which is arbitrarily chosen such that $|r_{m}|^2=95\%$ and $|t_{m}|^2 = 5\%$.

\begin{figure}[h!]
    \centering
    \begin{subfigure}[b]{0.49\textwidth}
        \includegraphics[width=\textwidth]{figures/fano_and_broadband_cavity_1000um.pdf}
        \caption{}
        \label{fig:standard_regime_trans}
    \end{subfigure}
    \begin{subfigure}[b]{0.49\textwidth}
        \includegraphics[width=\textwidth]{figures/fano_and_broadband_cavity_5um.pdf}
        \caption{}
        \label{fig:fano_regime_trans}
    \end{subfigure}
    \caption{In (a) is seen the comparison of broadband and single Fano cavity transmission spectra for a cavity length of $l \approx 1000 \mu m$, i.e. in the \emph{standard} regime. (b) shows the same comparison, but for a cavity length of $l \approx 5 \mu m$, i.e. in the so-called \emph{Fano} regime.}
\end{figure}

Figure \ref{fig:standard_regime_trans} shows the transmission spectra of the two cavities for a length of $l \approx 1000\mu m$, i.e. in the standard regime. It is clear from inspection of the figure that the resonance transmission profile of the standard broadband cavity is not wavelength dependent, in the sense that all fringes appear to have the same high finesse $\mathcal{F}$, i.e. ratio between the FSR and HWHM. This is not the case for the Fano cavity which is due to the wavelength dependence of the optical properties of the Fano mirror, as this causes the transmission and reflectivity to \emph{only} match those of the broadband mirror when on resonance. Furthermore, no significant difference in linewidth is seen for the transmission of the two cavities on resonance, as predicted by eq. (\ref{eq:analytical_linewidth}).

In figure \ref{fig:fano_regime_trans}, the transmission spectra of both cavities are shown for a length of $l\approx 5 \mu m$, i.e. in the Fano regime, where it is clearly seen that while the standard cavity experiences broadening for shorter cavity lengths\footnote{This is a natural consequence of the shorter cavity as this causes the lifetime inside the cavity to fall and hense the transmission HWHM to rise. The HWHM is inversly proportional to the lifetime and goes as $\delta \lambda \propto 1/\tau$.}, this is not the case for the Fano cavity transmission peak.

\begin{figure}[h!]
    \centering
    \begin{subfigure}[c]{0.64\textwidth}
        \centering
        \includegraphics[width=\textwidth]{figures/HWHM_broadband_vs_single_sim.pdf}
        \caption{}
        \label{fig:HWHM_broadband_vs_single_fano}
    \end{subfigure}
    \begin{subfigure}[c]{0.34\textwidth}
        \includegraphics[width=\textwidth]{figures/sim_single_vs_broadband_700um.pdf}
        \caption{}
        \label{fig:700um_broadband_and_single_fano_peak}
        \includegraphics[width=\textwidth]{figures/sim_single_vs_broadband_10um.pdf}
        \caption{}
        \label{fig:10um_broadband_and_single_fano_peak}
    \end{subfigure}
    \caption{(a) shows the approximate analytical resonance linewidths (eq. (\ref{eq:analytical_linewidth})) as a function of cavity length for the broadband and single Fano cavities together with linewidths of transmission profiles simulated using eq. (\ref{eq:single_fano_trans}) and eq. (\ref{eq:fabry_perot_trans}), found as parameters of least squares fits, for comparison. In (b) and (c) is seen transmission spectra of broadband and single Fano cavities of lengths $\sim 700 \mu m$ and $\sim 10 \mu m$, respectively. The spectra shown indicate each their respective linewidths, and are examples of the values plotted in (a).}
\end{figure}

Figure \ref{fig:HWHM_broadband_vs_single_fano} models the behavior of the linewidth of the single Fano cavity compared with the one for a broadband cavity of similar optical properties, as a function of wavelength. Here it is easily seen where the linewidth of the single Fano transmission begins to saturate, and hence deviate from the one of the broadband cavity. The plotted line in the figure is calculated using eq. (\ref{eq:analytical_linewidth}) while the points depicit linewidths found as a fitting parameter from a least squares fit of the general Fano model in eq. (\ref{eq:general_fano_model}) to transmission spectra simulated by the Fabry-Perot (eq. (\ref{eq:fabry_perot_trans})) and single Fano (eq. (\ref{eq:single_fano_trans})) transmission functions. Finally, it can be concluded that the approximate analytical expression for the linewidth of the broadband and single fano cavities in eq. (\ref{eq:analytical_linewidth}) correlates very well with the values found from the simulated spectra.

\subsection{The double Fano cavity}

\subsubsection{The double Fano cavity model}

While the single Fano cavity is usually known in the appropriate litterature as simply a \emph{Fano cavity}, I have insisted on including the fact that it contains only one Fano mirror, and hence denoted it the \emph{single} Fano cavity. This addition is justified by the contents of this section, and namely that we now move on to the \emph{double} Fano cavity, which as the name suggests consists of two Fano mirrors. The schematics of this configuration is shown together with the one for the single Fano cavity in figure \ref{fig:single_and_double_fano_sketch}.

\begin{figure}[h!]
    \centering
    \begin{subfigure}[b]{0.3\textwidth}
        \includegraphics[width=\textwidth]{figures/single_fano_sketch.pdf}
        \caption{}
        \label{fig:single_fano_sketch2}
    \end{subfigure}
    \hspace{1cm}
    \begin{subfigure}[b]{0.3\textwidth}
        \includegraphics[width=\textwidth]{figures/double_fano_sketch.pdf}
        \caption{}
        \label{fig:double_fano_sketch}
    \end{subfigure}
    \caption{(a) shows schematics of the single Fano cavity consisting of a broadband mirror with transmission and reflectivity coefficients $t_m$, $r_m$ and a Fano mirror with coefficients $t_g$, $r_g$. (b) shows the schematics of the double Fano cavity consisting of two Fano mirrors with coefficients $t_g$, $r_g$, $t_g^{\prime}$, $r_g^{\prime}$. Both cavities are separated by a cavity length $l$.}
    \label{fig:single_and_double_fano_sketch}
\end{figure}

Here it is evident that instead of having one set of reflectivity and transmission coefficients that depend on the incident wavelength, we now have two. In order to model the transmission of the double Fano cavity, we once again consider the transmission function for the normal incident and planar Fabry-Perot cavity in eq. (\ref{eq:fabry_perot_trans}), this time with $r,t \rightarrow r_g(\lambda),t_g(\lambda)$ and $r^{\prime},t^{\prime} \rightarrow r_g^{\prime}(\lambda),t_g^{\prime}(\lambda)$\cite{Naesby}. We rewrite the Fabry-Perot transmission function with the addressed substitutions of the optical coefficients and such that it decribes the normalized transmission amplitudes $T_{cav} = |E_{out}|^2/|E_{0,in}|^2$ and get
\begin{equation}
    T_{cav} = \frac{t_g(\lambda) t_g^{\prime}(\lambda) e^{i\phi}}{1 - r_g(\lambda)r_g^{\prime}(\lambda) e^{2 i \phi}}.
    \label{eq:double_fano_transmission}
\end{equation}
The subscript $g$ indicates a grating transmission or reflectivity. Figure \ref{fig:double_fano_transmission} shows an example of the normalized transmission spectrum of an $\sim 30 \mu m$ double Fano cavity on- and off-resonance, found using eq. (\ref{eq:double_fano_transmission}).

\begin{figure}[h!]
    \centering
    \begin{subfigure}[c]{0.49\textwidth}
        \centering
        \includegraphics[width=\textwidth]{figures/double_fano_full_range_30um.pdf}
        \caption{}
        \label{fig:double_full_range}
    \end{subfigure}
    \begin{subfigure}[c]{0.49\textwidth}
        \includegraphics[width=\textwidth]{figures/double_fano_short_range_30um.pdf}
        \caption{}
        \label{fig:double_short_range}
    \end{subfigure}
    \caption{An example of a normalized double Fano transmission spectrum of an $\sim 30 \mu m$ cavity. (a) shows a long-range wavelength scan, depicting both the on- and off-resonance behaviour of the double Fano cavity transmission, while (b) shows the transmission focussed specifically around the resonance position. Both examples are shown together with the reflectivity of the Fano mirror used to model them.}
    \label{fig:double_fano_transmission}
\end{figure}

\subsubsection{Transmission linewidth}

In order to describe the analytical linewidth of the transmisison profile of the double Fano cavity, we take a similar approach as in the single Fano case outlined in section \ref{sec:single_fano_cavity_trans_linewidth}. It can be shown from eq. (\ref{eq:analytical_linewidth}) when including the wavelength dependence of the optical coefficients of both Fano mirrors, that the HWHM $\delta \lambda$ of the double Fano cavity transmission profile is approximately given as
\begin{equation}
    \delta \lambda^{double} \approx \frac{1}{\frac{1}{\delta \lambda_c} + \frac{1}{\delta \lambda_g^{double}}},
    \label{eq:analytical_linewidth_double}
\end{equation}
where 
\begin{equation}
    \delta \lambda_c = \frac{\lambda_0^2}{8 \pi l} (|t_g(\lambda_0)|^2 + |t_g^{\prime}(\lambda_0)|^2 + L)
\end{equation}
is still the HWHM of a broadband cavity and
\begin{equation}
    \delta \lambda_g^{double} = \frac{\gamma \lambda}{4 (1-r_d)}(|t_g(\lambda_0)|^2 + |t_g^{\prime}(\lambda_0)|^2 + L).
    \label{eq:double_fano_linewidth_contribution}
\end{equation}
is the HWHM of the double Fano cavity in the Fano regime. Note that $\delta \lambda^{double} = \delta \lambda^{single}/2$ for $l \rightarrow 0$ when eq. (\ref{eq:analytical_linewidth_double}) is predominantly given by eq. (\ref{eq:double_fano_linewidth_contribution}). In this brief evaluation of the estimated analytical linewidth of the double Fano transmission profile it is assumed that all defining parameters of the two Fano mirrors are identical, except for the cavity and guided-mode resonance wavelengths $\lambda_{0,1}$. Namely the following relevant parameters are assumed identical,
\begin{equation}
    r_d = r_d^{\prime} \: \text{ and } \: \gamma \lambda = \gamma \lambda^{\prime}.
\end{equation}
In this way any spectral detuning of the two Fano mirrors used to make a cavity is included in the analytical expression. The spectral detuning, and the effect hereof, will be further described in section \ref{sec:spectral_detuning}.

\subsubsection{Single and double Fano cavity comparison}

Using the analytical expression for the double Fano cavity transmission in eq. (\ref{eq:analytical_linewidth_double}) we are now in a position to compare the single and double Fano cavities. Note that we at this point only consider the ideal case of the double Fano cavity where additional cavity losses are neglected and the two Fano mirrors used are identical, i.e. the cavity is said to be \emph{symmetrical}. The additional cavity losses are explicitly set to be given as
\begin{equation}
    L = 1-|r_g|^2-|t_g|^2 = 0.
\end{equation}

\begin{figure}[h!]
    \centering
    \begin{subfigure}[c]{0.49\textwidth}
        \centering
        \includegraphics[width=\textwidth]{figures/single_and_double_1000um.pdf}
        \caption{}
        \label{fig:double_in_standard_regime}
    \end{subfigure}
    \begin{subfigure}[c]{0.49\textwidth}
        \includegraphics[width=\textwidth]{figures/single_and_double_5um.pdf}
        \caption{}
        \label{fig:double_in_fano_regime}
    \end{subfigure}
    \caption{Comparison of the single and double Fano transmission spectra in the \emph{standard} and \emph{Fano} regimes. (a) shows the spectral comparison for an $\sim 1000 \mu m$ cavity, i.e. in the \emph{standard} regime, while (b) shows the same for an $\sim 5 \mu m$ cavity, i.e. in the \emph{Fano} regime. The Fano mirror reflectivity is depicted in both (a) and (b).}
    \label{fig:double_in_standard_and_fano_regimes}
\end{figure}

Figure \ref{fig:double_in_standard_and_fano_regimes} shows the transmission of the ideal double Fano cavity and the corresponding single Fano cavity for comparison. 

Figure \ref{fig:double_in_standard_regime} shows the transmission for a cavity length of $\sim 1000 \mu m$ which is well-inside the standard regime outlined in section \ref{sec:single_fano_cavity_trans_linewidth} where the standard broadband and single Fano cavities produce transmission spectra of roughly identical linewidths. In this regime, due to the $1/l$ proportionality of the FSR, the off-resonance behavior of the double Fano cavity transmission is visible in the range plottet. It is seen, contrary to the single Fano cavity, that the transmission at each cavity resonance reaches a normalized transmission of $|E_{out}|^2/|E_{0,in}|^2=1$. This is due to the fact that the two gratings, while they have wavelength dependent transmission and reflectivity coefficients, always have identical ones for the ideal case, which maximizes the Fabry-Perot transmission function and ensures unity transmission at any cavity resonance. The minimum level of the cavity changes according to the reflectivity of the Fano mirrors and by that the HWHM also changes as we move further from the guided-mode resonance. Both the minimum transmission level and the HWHM converge when moving away from the resonance wavelength, when the reflectivity becomes constant. 

Figure \ref{fig:double_in_fano_regime} shows the transmission of a double Fano cavity of length $\sim 5 \mu m$ which, contrary to the one in figure \ref{fig:double_in_standard_regime}, is well within the Fano regime. It is seen that the double Fano cavity transmission produces a resonance peak with a HWHM narrower than the one for the single Fano cavity, as is predicted in eq. (\ref{eq:double_fano_linewidth_contribution}). Furthermore, the immediate off-resonance behavior of the double Fano cavity transmission in the Fano regime, is drastically different than for the single Fano cavity. This is due to the collective higher transmission in this region compared with the single Fano case where the broadband mirror has a constant, and often high, reflectivity and hence a correspondingly low transmission.

\begin{figure}[h!]
    \centering
    \begin{subfigure}[c]{0.64\textwidth}
        \centering
        \includegraphics[width=\textwidth]{figures/HWHM_broadband_vs_single_vs_double_sim.pdf}
        \caption{}
        \label{fig:HWHM_double_vs_single_vs_broadband}
    \end{subfigure}
    \begin{subfigure}[c]{0.34\textwidth}
        \includegraphics[width=\textwidth]{figures/sim_single_vs_double_700um.png}
        \caption{}
        \label{fig:700um_double_and_single_fano_peak}
        \includegraphics[width=\textwidth]{figures/sim_single_vs_double_10um.png}
        \caption{}
        \label{fig:10um_double_and_single_fano_peak}
    \end{subfigure}
    \caption{(a) shows the analytical resonance linewidths (eqs. (\ref{eq:analytical_linewidth}), (\ref{eq:analytical_linewidth_double}), and (\ref{eq:analytical_linewidth_broadband})) as a function of the cavity length for the broadband, single and double Fano cavities together with linewidths of corresponding profiles simulated using eqs. (\ref{eq:single_fano_trans}), (\ref{eq:double_fano_transmission}), and (\ref{eq:fabry_perot_trans}) for comparison. In (b) and (c) is seen intracavity spectra of single and double Fano cavities of lengths $\sim 700 \mu m$, and $\sim 10 \mu m$, respectively. The spectra shown indicate each their respective linewidths, and are examples of the values plotted in (a). Note however, that the broadband cavity peak has been left out of (b) and (c).}
\end{figure}

Figure \ref{fig:HWHM_double_vs_single_vs_broadband} shows the analytical linewidth calculated and compared for the broadband, single Fano, and double Fano cavities, calculated using eqs. (\ref{eq:analytical_linewidth_broadband}), (\ref{eq:analytical_linewidth}), (\ref{eq:analytical_linewidth_double}). These are compared with linewidths found as fitting parameters from a least squares fit of the general Fano transmission formula given in eq. (\ref{eq:general_fano_model}) to transmission spectra calculated using eqs. (\ref{eq:fabry_perot_trans}), (\ref{eq:single_fano_trans}), and (\ref{eq:double_fano_transmission}). According to eq. (\ref{eq:analytical_linewidth_double}) the linewidth of the double Fano cavity transmission should converge to exactly half that of the single Fano cavity, meaning that
\begin{equation}
    \delta \lambda_{double} = \frac{\delta \lambda_{single}}{2}, \hspace{0.5cm} \text{for } l \rightarrow 0.
\end{equation}
This is supported well by the simulation depicted in figure \ref{fig:HWHM_double_vs_single_vs_broadband}.

Figures \ref{fig:700um_double_and_single_fano_peak} and \ref{fig:10um_double_and_single_fano_peak} contain examples of intracavity\footnote{The reason these are shown as intracavity spectra instead of transmission spectra is due to the immediate off-resonance behavior of the double Fano cavity transmission. The shape makes accurate fitting a challenge, and while the intracavity spectra cannot be measured in the lab, the linewidth when simulated is identical to the resulting transmission peaks.} spectra of single and double Fano cavities and corresponding least squares fits to the general Fano model in order to determine their linewidths. 

\subsubsection{Additional cavity losses}

Thus far we have only examined a lossless double Fano cavity where
\begin{equation}
    |r_g|^2 + |t_g|^2 = 1
\end{equation}
is fulfilled for both Fano mirrors used to contruct the cavity.

\begin{figure}[h!]
    \centering
    \includegraphics[width=0.6\textwidth]{figures/double_fano_loss_scan.pdf}
    \caption{Resonant transmission spectra for different values of the additional loss term $L$, of a symmetric double Fano cavity of length $\sim 30 \mu m$.}
    \label{fig:double_loss_scan}
\end{figure}

In this section we will investigate what happens when we introduce additional cavity losses\footnote{Not to be confused with the often used definition of cavity losses given as $L=1-|r|^2$ where anything not being reflected back into the cavity is considered to have a contribution to the losses.}
\begin{equation}
    L = 1 - |r_g|^2 - |t_g|^2,
\end{equation}
leading to the modified condition
\begin{equation}
    |r_g|^2 + |t_g|^2 + L = 1.
\end{equation}

Figure \ref{fig:double_loss_scan} shows double Fano transmission spectra for a symmetric cavity with varying values for the additional cavity losses. It is readily seen that the maximum value reached in each spectrum is rapidly reduced with the introduction of these losses. 

Since the linewidth is defined as the HWHM of the transmission profile, this will naturally vary as a function of additional cavity losses. This is depicted in figure \ref{fig:HWHM_vs_losses} where the HWHM is shown for different values of $L$. Examples of intracavity spectra taken for different values of $L$ are shown in figures \ref{fig:HWHM_vs_losses_whole_figure}b and \ref{fig:HWHM_vs_losses_whole_figure}c, along with their respective least squares fits to the general Fano model in eq. (\ref{eq:general_fano_model}) and linewidths found as fitting parameters hereof. 

\begin{figure}[h!]
    \centering
    \begin{subfigure}[c]{0.64\textwidth}
        \centering
        \includegraphics[width=\textwidth]{figures/linewidth_vs_losses.pdf}
        \caption{}
        \label{fig:HWHM_vs_losses}
    \end{subfigure}
    \begin{subfigure}[c]{0.34\textwidth}
        \includegraphics[width=\textwidth]{figures/double_2_loss_30um.pdf}
        \caption{}
        \includegraphics[width=\textwidth]{figures/double_24_loss_30um.pdf}
        \caption{}
    \end{subfigure}
    \caption{(a) shows the linewidth (HWHM) of a symmetric double Fano cavity of length $\sim 30 \mu m$  as a function of additional cavity losses $L = 2(|r_g|^2 - |t_g|^2)$. Each point is found as a fitting parameter of a least squares fit of the intracavity double Fano spectrum for a certain value of $L$ to the general Fano model. The plotted line indicates the analytical value of the linewidth (eq. \ref{eq:analytical_linewidth}) as a function of $L$, for comparison. In (b) and (c) are seen examples of intracavity double Fano spectra, with their respective linewidths, for cavities of $L=2\%$ and $L=24\%$, respectively.}
    \label{fig:HWHM_vs_losses_whole_figure}
\end{figure}

Figure \ref{fig:HWHM_vs_losses} shows the simulated linewidths of intracavity spectra as a function of total cavity losses $L$, and is compared with the analytical formula for the double Fano linewidth in eq. (\ref{eq:analytical_linewidth_double}). It is seen that the analytical model proves, once again, to be a powerful predicting technique for the linewidth, and especially for cavities with relatively low losses. 

\subsubsection{Spectral detuning (lossless)} \label{sec:spectral_detuning}

Up until this point, it has been assumed that the Fano mirrors making up the double Fano cavity has been identical, namely the cavity has been \emph{symmetric}. However, in practice this is very unlikely to be the case, as any Fano mirror constructed is bound to have some uncertainties attached to the physical parameters describing it. For that reason we investigate the effect of an \emph{assymetric} cavity on the resulting transmission profile. Here we remember that each Fano mirror is describes by a set of parameters, $\lambda_0$, $\lambda_1$, $t_d$, $\gamma \lambda$ and $\beta$, which respectively describe the cavity resonance wavelength, guided-mode resonance wavelength, direct transmission, guided-mode resonance linewidth and additional losses of each grating. In order to model only a spectral detuning, we therefore simply change the parameters regarding the cavity and guided-mode resonance wavelengths, $\lambda_0$ and $\lambda_1$ by an amount corresponding to the detuning we wish to study. For this section the parameters will be given as
\begin{equation}
    \begin{split}
    \lambda_0 = &951 nm,\:\: \lambda_1 = 951 nm,\:\: t_d = 80\%,\\ &\gamma \lambda = 0.5 nm\: \text{ and }\: \beta = 10^{-6}
    \end{split}
    \label{eq:grating_params}
\end{equation}
for the unchanged Fano mirror, and
\begin{equation}
    \begin{split}
    \lambda_0^{\prime} = &\lambda_0 + \Delta,\:\: \lambda_1^{\prime} = \lambda_1 + \Delta,\:\: t_d^{\prime} = t_d,\\ &\gamma \lambda^{\prime} = \gamma \lambda\: \text{ and }\: \beta^{\prime} = \beta
    \end{split}
    \label{eq:detuned_grating_params}
\end{equation}
for the \emph{detuned} Fano mirror. Figure \ref{fig:detuned_grating_spectra} shows the normalized reflectivity and transmission spectra of the Fano mirrors described by the parameters in eqs. (\ref{eq:grating_params}) and (\ref{eq:detuned_grating_params}).

\begin{figure}[h!]
    \centering
    \includegraphics[width=0.7\textwidth]{figures/detuned_grating_spectra.pdf}
    \caption{The normalized reflectivity and transmission spectra of the Fano mirrors described in eqs. (\ref{eq:grating_params}) and (\ref{eq:detuned_grating_params}) for detuning $\Delta = 0.3nm$.}
    \label{fig:detuned_grating_spectra}
\end{figure}

As has been observed in previous sections, the Fano resonance transmission peak is present at the point where the grating reflectivity $r_g(\lambda)$ is maximized and transmission $t_g(\lambda)$ minimized. However, when $\lambda_0 \neq \lambda_0^{\prime}$ and $\lambda_1 \neq \lambda_1^{\prime}$, this is no longer a trivial conclusion to draw. The question of whether the cavity resonance should be tuned to match the guided-mode resonance wavelength of one grating or the other, or maybe somewhere in between does not have an obvious answer. This will be further expanded upon later in section \ref{sec:spacial_detuning}, but in order to move forward with the investigation of the spectral detuning we, for now, accept that the optimal cavity length, is the one corresponding to a cavity resonance $\lambda_t$ given, exactly between the two guided-mode resonance wavelengths, as 
\begin{equation}
    \lambda_t = \frac{\lambda_{0,1} + \lambda_{0,1}^{\prime}}{2}.
    \label{eq:transmission_wavelength}
\end{equation}
The subscribt \emph{0,1} indicate that the two are interchangable, as the two are assumed identical at this time, and \emph{t} is for \emph{transmission} as this is, later on, to be used experimentally as the \emph{transmission wavelength}. 

\begin{figure}[h!]
    \centering
    \includegraphics[width=0.7\textwidth]{figures/detuned_full_range_double_trans.pdf}
    \caption{The double Fano transmission spectra of a cavity detuned by $\Delta = 0.3nm$, as seen in figure \ref{fig:detuned_grating_spectra}, together with the reflectivity and transmission spectra of the Fano mirrors used for the simulation.}
    \label{fig:detuned_double_fano_transmission}
\end{figure}

Figure \ref{fig:detuned_double_fano_transmission} shows the transmission spectrum of a detuned double Fano cavity with parameters corresponding to the transmission and reflectivity spectra in figure \ref{fig:detuned_grating_spectra}, meaning that $\Delta = 0.3nm$. The transmission wavelength $\lambda_t$ is chosen such that eq. (\ref{eq:transmission_wavelength}) is fulfilled, and correspondingly it is seen that the transmission peak is placed exactly between the maximum (minimum) reflectivites (transmissions) of the two grating, i.e. between the two guided-mode resonance wavelengths. Furthermore, it can be concluded that the detuning is chosen such that the overlap between the guided-mode resonances is still substantiel enough for them to couple and hence for the the overall Fano resonance to be excited. 

While knowing that a detuning of $0.3nm$ is acceptable in terms of exciting the Fano resonance in the lossless case is a nice result, it does not provide much in terms of estimating the acceptable detuning for any experimental purposes. In figure \ref{fig:detuning_scan} the double Fano transmission is shown for increasing detuning $\Delta$ and transmission wavelength $\lambda_t = (\lambda_{0,1} + \lambda_{0,1}^{\prime})/2$. 

\begin{figure}[h!]
    \centering
    \begin{subfigure}[b]{0.49\textwidth}
        \includegraphics[width=\textwidth]{figures/detuning_scan_double_fano_30um.pdf}
        \caption{}
        \label{fig:detuning_scan}
    \end{subfigure}
    \begin{subfigure}[b]{0.49\textwidth}
        \includegraphics[width=\textwidth]{figures/detuning_scan_intracavity_double_fano_30um.pdf}
        \caption{}
        \label{fig:intracavity_detuning_scan}
    \end{subfigure}
    \caption{In (a) is seen lossless double Fano transmission spectra for increasing value of the detuning $\Delta = |\lambda_{0,1} - \lambda_{0,1}^{\prime}|$. It is readily seen that the transmission wavelength $\lambda_t = (\lambda_{0,1} + \lambda_{0,1}^{\prime})/2$ goes up, and that the linewidth likewise becomes larger with increasing detuning. The spectrum for $\Delta = 1.2nm$ shows the effect of the detuning becoming too large for the guided-modes of the two Fano mirrors to have substantial overlap and thus to sustain the Fano resonance. (b) shows the intracavity spectra corresponding to the transmission spectra in (a).}
    \label{fig:detuning_threshold_lossless}
\end{figure}

It is readily seen that with increasing positive detuning, relative to the resonant wavelength of the unchanged Fano mirror, the peak shifts to higher wavelengths, which in itself is trivial from eq. (\ref{eq:transmission_wavelength}). The linewidth is also seen to increase with the detuning, and the peak eventually collapses when the overlap between the guided-mode resonance profiles of the two Fano mirrors becomes too small to sustain the Fano resonance. Figure \ref{fig:intracavity_detuning_scan} shows the intracavity spectra corresponding to the transmission spectra in figure \ref{fig:detuning_scan}, and provides valuable insight into the mode density inside the cavity for the different values of $\Delta$. It is clearly demonstrated by the two figures that the spectral overlap is a very crucial parameter of the double Fano cavity, and is paramount in describing the cavity's ability to sustain the Fano resonance modes. 

\subsubsection{Spacial detuning (lossless)} \label{sec:spacial_detuning}

As mentioned above in section \ref{sec:spectral_detuning}, any spectral detuning gives rise to the potential of a spacial detuning as $2l = m\lambda$ (eq. (\ref{eq:general_cavity_condition})) must be fulfilled for any sustained plane-wave mode inside a normal incident optical cavity. We denote the length corresponding to the resonance wavelength of the unchanged Fano mirror as simply $l$, while the one for the detuned Fano mirror will be denoted as $l^{\prime}$. Previously we assumed that the optimal length of a detuned double Fano cavity is the one where the so-called transmission wavelength is given as 
\begin{equation}
    \lambda_t = \frac{\lambda_{0,1} + \lambda_{0,1}^{\prime}}{2}.
\end{equation}
And while this does turn out to be the case, we must inverstigate the resonant length-dependence of the linewidth of the Fano resonance profile in order to arrive at this conclusion indefinitely. 

\begin{figure}[h!]
    \centering
    \begin{subfigure}[b]{0.49\textwidth}
        \includegraphics[width=\textwidth]{figures/small_detuning_length_scan_short.pdf}
        \caption{}
        \label{fig:detuned_small_length_scan_long}
    \end{subfigure}
    \begin{subfigure}[b]{0.49\textwidth}
        \includegraphics[width=\textwidth]{figures/small_detuning_length_scan_long.pdf}
        \caption{}
        \label{fig:detuned_small_length_scan_short}
    \end{subfigure}
    \caption{(a) shows lossless long-range transmission spectra of double Fano cavities of lengths $l \rightarrow l^{\prime} \approx 30 \mu m$ with $\Delta = 0.1nm$. (b) shows the same spectra as seen in (a), zoomed in on the range around the transmission wavelength $\lambda_t$.}
    \label{fig:small_detuning_length_scans}
\end{figure}

Figure \ref{fig:small_detuning_length_scans} shows double Fano resonance transmission profiles plottet for varying cavity length $l \rightarrow l^{\prime}$ with detuning $\Delta = 0.1nm$. Looking at figure \ref{fig:detuned_small_length_scan_long} we see the long range scan of the transmission profiles where there still appears to be a substantial overlab between the guided-modes of the two Fano mirrors. The main differences are seen as changes in the intensity in the immediate off-resonance regime. This is due to the dramatic wavelength (and thus length) dependence in the areas close to the resonance position, which makes even very small changes visible on the spectra.

Figure \ref{fig:detuned_small_length_scan_short} shows the same spectra, but enhanced around the resonance wavelength, and this too shows that the substantiel guided-mode overlap results in little visible change, except for shifting the peak position, when scanning between the resonant cavity lenghts. 

\begin{figure}[h!]
    \centering
    \begin{subfigure}[b]{0.49\textwidth}
        \includegraphics[width=\textwidth]{figures/detuned_length_scan_long.pdf}
        \caption{}
        \label{fig:detuned_length_scan_long}
    \end{subfigure}
    \begin{subfigure}[b]{0.49\textwidth}
        \includegraphics[width=\textwidth]{figures/detuned_length_scan_short.pdf}
        \caption{}
        \label{fig:detuned_length_scan_short}
    \end{subfigure}
    \caption{(a) shows lossless long-range transmission spectra of double Fano cavities of lengths $l \rightarrow l^{\prime} \approx 30 \mu m$ with $\Delta = 0.3nm$. (b) shows the same spectra as seen in (a), zoomed in on the range around the transmission wavelength $\lambda_t$.}
    \label{fig:mid_detuning_length_scans}
\end{figure}

Figure \ref{fig:mid_detuning_length_scans} shows similar transmission profiles as in figure \ref{fig:small_detuning_length_scans}, only with an increased detuning of $\Delta = 0.3nm$. Looking at the long range scan in figure {\ref{fig:detuned_length_scan_long}} and comparing with the one in figure \ref{fig:detuned_small_length_scan_long}, it is seen that the displacement of the off-resonance Fabry-Perot like fringes have increased with the slightly higher detuning. This is not surprising as the Fabry-Perot cavity modes, like any other interference pattern inside a cavity must fulfill the brightness identity $2l = m \lambda$. Furthermore, the immediate off-resonance regime shows a slight increase in the intensity displacement when compared with the less detuned example. 

Figure \ref{fig:detuned_length_scan_short}, which shows the spectra enhanced around the resonance position, provides a more detailed image of what happens to the transmission profiles as a function of the cavity length. Here it is seen that the peaks at the edges of the length interval have varied in both position and height, much more dramatically than the once in figure \ref{fig:detuned_small_length_scan_short}. In short, the increased detuning has also increased the fractional length-dependence of the double Fano cavity\footnote{It must be noted here that the range in which the cavity length is scanned is increased with the detuning. So the length-dependence is fractional in the sense that the step size is defined by the size of the interval.}, as the resonance transmission profile seem to change \emph{more} when varying the length of the cavity between the guided-mode resonance lengths, in this case.

\begin{figure}[h!]
    \centering
    \begin{subfigure}[b]{0.49\textwidth}
        \includegraphics[width=\textwidth]{figures/large_detuning_length_scan_long.pdf}
        \caption{}
        \label{fig:detuned_large_length_scan_long}
    \end{subfigure}
    \begin{subfigure}[b]{0.49\textwidth}
        \includegraphics[width=\textwidth]{figures/large_detuning_length_scan_short.pdf}
        \caption{}
        \label{fig:detuned_large_length_scan_short}
    \end{subfigure}
    \caption{(a) shows lossless long-range transmission spectra of double Fano cavities of lengths $l \rightarrow l^{\prime} \approx 30 \mu m$ with $\Delta = 0.8nm$. (b) shows the same spectra as seen in (a), zoomed in on the range around the transmission wavelength $\lambda_t$.}
    \label{fig:large_detuning_length_scans}
\end{figure}

Figure \ref{fig:large_detuning_length_scans} shows the effect of a relatively large detuning of $\Delta = 0.8nm$. This is simply included for a visual representation of a detuning which is considered "too large" for any practical use. The long range scan in figure \ref{fig:detuned_large_length_scan_long} shows that the Fabry-Perot like fringes are now too displaced to be compared in terms of their positions, and figure \ref{fig:detuned_large_length_scan_short} showing the spectra enhanced around the resonance wavelength shows roughly the same trend as in figure \ref{fig:detuned_length_scan_short}, only that this example is greatly broadened in comparison. It is though noted, that while broadened, the Fano resonance mode is sustained even for the relatively large detuning.

In order to get a clear and qualitative picture of the double Fano cavity transmission profile as a function of both cavity length and wavelength, we visualize the varying of both in a heat map and let the color indicate the transmission intensity. This is shown in figure \ref{fig:cmap_single} which depicts a lossless cavity of length $l\approx 20 \mu m$ and detuning $\Delta = 0.3nm$. Here the movement of the resonance peak as a function of the cavity length is clearly seen as a slope of the "line" representing the high intensity region due to the Fano resonance. 

\begin{figure}[h!]
    \centering
    \begin{subfigure}[c]{0.49\textwidth}
        \centering
        \includegraphics[width=\textwidth]{figures/cmap_with_slice_indicators1.pdf}
        \caption{}
        \label{fig:cmap_single}
    \end{subfigure}
    \begin{subfigure}[c]{0.49\textwidth}
        \centering
        \includegraphics[width=0.49\textwidth]{figures/cmap_slice2.pdf}
        \newline
        \includegraphics[width=0.49\textwidth]{figures/cmap_slice1.pdf}
        \includegraphics[width=0.49\textwidth]{figures/cmap_slice3.pdf}
        \caption{}
        \label{fig:cmap_slices}
    \end{subfigure}
    \caption{(a) shows a heat map of the lossless double Fano cavity transmission as a function of wavelength and cavity lengths ranging $l \rightarrow l^{\prime} \approx 20 \mu m$ for $\Delta = 0.3nm$. (b) show slices of the heat map for comparing different cavity lenghts, and are indicated by their respective color and linetype in (a). The respecitive cavity lengths and linewidths are given as: $l_{magenta} = (0.2l+0.8l^{\prime})$, $HWHM_{magenta} = 71.3pm$, $l_{cyan} = (0.8l+0.2l^{\prime})$, $HWHM_{cyan} = 71.3pm$ and $l_{lime} = (l+l^{\prime})/2$, $HWHM_{lime} = 58.6pm$.}
    \label{fig:cmap_and_slices}
\end{figure}

Considering only the heat map, it is not easily visible which cavity length is the optimal one, however, the \emph{magenta}, \emph{cyan}, and \emph{limegreen} lines across the heat map indicate slices which are depicted separately in figure \ref{fig:cmap_slices}. It is seen by analysing the transmission profiles of these cavities of specific lengths, how they vary in both position and linewidth. The resonance wavelength positions are given relative to the cavity lengths chosen. The lengths and corresponding linewidths are given as
\begin{equation}
    \begin{split}
        l_{magenta} = 0.2 l + 0.8 l^{\prime} &\rightarrow HWHM_{magenta} = 71.3pm\\ l_{cyan} = 0.8 l + 0.2 l^{\prime} &\rightarrow HWHM_{cyan} = 71.3pm\\ l_{lime} = (l + l^{\prime})/2 &\rightarrow HWHM_{lime} = 58.6pm.
    \end{split}
\end{equation}
It turns out that the linewidths of the cyan and magenta transmission profiles are identical\footnote{This is only exactly correct because the only varied parameter of the two Fano mirrors are the cavity- and guided-mode-resonant wavelengths $\lambda_0$ and $\lambda_1$. In a more realistic case where several parameters vary, the attributes of the transmission profiles are unlikely to act fully symmetrically when varying the length $l \rightarrow l^{\prime}$.}, while the limegreen transmission profile seem to have a narrower profile at resonance. 

\begin{figure}[h!]
    \centering
    \includegraphics[width=0.6\textwidth]{figures/cmap_lw_vs_l_intracavity.pdf}
    \caption{Linewidth as a function of cavity length $l \rightarrow l^{\prime}$. Each point is found as a fitting parameter from a least squares fit of the intracavity lossless double Fano spectrum for each respective cavity length, to the generalized Fano model in eq. (\ref{eq:general_fano_model}).}
    \label{fig:lw_vs_l_scan}
\end{figure}

This trend is further examined in figure \ref{fig:lw_vs_l_scan} where the linewidths of intracavity spectra are shown as a function of the cavity length. The parameters used are the same as in figure \ref{fig:cmap_and_slices} and clearly indicates that the optimal cavity length of a detuned double Fano cavity is the one fulfilling 
\begin{equation}
    \lambda_t = \frac{\lambda_{0,1} + \lambda_{0,1}^{\prime}}{2},
\end{equation}
for the transmission wavelength $\lambda_t$, and thus corresponding cavity length $l_t$.

As a visual and qualitative representation of the effect of increasing the detuning $\Delta$, figure \ref{fig:cmaps_detuning} shows heat maps similar to the one in figure \ref{fig:cmap_and_slices}, but for a range of values for the detuning $\Delta=0.1nm \rightarrow \Delta=0.9nm$. It is readily seen that the aforementioned slope of the high intensity region, indicating the resonance peak, increases with the detuning. This is a representation of the peak moving to higher wavelengths, both for the optimal transmission wavelength and for the one closer to the detuned Fano mirror guided-mode resonance. The broadening of the peak is also displayed in a way that is, while only qualitative, convincing. 

\begin{figure}[h!]
    \centering
    \includegraphics[width=0.8\textwidth]{figures/cmaps_detuning_30um.pdf}
    \caption{Series of heat maps showing lossless double Fano transmission spectra as a function of the cavity length, for increasing detuning $\Delta$. The movement of the resonant wavelength is shown as the increasing slope of the high intensity region representing the resonance, for increasing detuning $\Delta$. The broadening as a function of detuning is easily seen as it is qualitatively displayed in the series of figures until the spectral overlap makes the cavity inable of sustaining the Fano resonance.}
    \label{fig:cmaps_detuning}
\end{figure}

The last image in figure \ref{fig:cmaps_detuning}, displays the case of $\Delta=0.9nm$. Here it is seen that the spectral overlap has become too small to sustain the Fano resonance inside the cavity, and the intensity at the, previously, optimal transmission wavelength is therefore at a minimum.  If however, we look at the areas of the heat map corresponding to cavity lengths of exactly $l$ or $l^{\prime}$ and wavelengths $\lambda_{0,1}$ or $\lambda_{0,1}^{\prime}$, we see that there is a slight increase in the transmission intensity. This is due to the fact that the cavity, in this position, corresponds to a single Fano cavity with a low finesse, as one of the Fano mirrors are exactly on resonance while the other is so far away that it acts as simply a poor "broadband" mirror. 

